\documentclass[9pt,a4paper,twocolumn]{article}
\usepackage[utf8]{inputenc}
\usepackage[T1]{fontenc}
\usepackage[greek,french]{babel}
\usepackage{hyperref}
\usepackage{amssymb}
\usepackage{verbatim}
\usepackage{xargs}

% \usepackage[pagewise]{lineno}
% \linenumbers

%\usepackage{setspace}
%\setstretch{1,5}

%%%%%%%%%%%%%%%%%%%%%%%%%%%%%
%%%%    Pour les graphes %%%%
%%%%%%%%%%%%%%%%%%%%%%%%%%%%%
\usepackage{tikz,pgf}
\usetikzlibrary{arrows,automata} % Pour les graphe
\tikzset{every picture/.style={execute at begin picture={%
   \shorthandoff{:;!?};}
}}
\usepackage{pgf-umlsd}
\usepackage{minted}

%%%%%%%%%%%%%%%%%%%%%%%%%%%%%%%%%
%%%%     Pour les figures    %%%%
%%%%%%%%%%%%%%%%%%%%%%%%%%%%%%%%%
\usepackage{epsfig}
%\usepackage{slashbox}

%%%%%%%%%%%%%%%%%%%%%%%%%%%%%%%%%%%%%%%%%%%%%
%%%%    Pour les théorêmes et tout...    %%%%
%%%%%%%%%%%%%%%%%%%%%%%%%%%%%%%%%%%%%%%%%%%%%
\usepackage{amsmath,amsthm}
\theoremstyle{definition}
\newtheorem{theorem}{Th\'eor\`eme}
\newtheorem{hypo}{Hypoth\`ese}
\newtheorem{definition}{Définition}
\newtheorem{lemma}{Lemme}
\newtheorem{hyp}{Hypoth\`ese}
\usepackage{stmaryrd}
\usepackage{mathtools}

\DeclarePairedDelimiter\ceil{\lceil}{\rceil}
\DeclarePairedDelimiter\floor{\lfloor}{\rfloor}

% use biblatex for bib
%\usepackage{biblatex}
%\addbibresource{biblio.bib}

\usepackage{multirow}

\usepackage{caption}
\captionsetup[table]{name=Tableau}

%%%%%%%%%%%%%%%%%%%%%%%%%%%%%%%%%%%%%%%%%%%%
%%%%    Pour une belle mise en page     %%%%
% %%%%%%%%%%%%%%%%%%%%%%%%%%%%%%%%%%%%%%%%%%%%
% Grosses marges par défaut
%\usepackage{geometry}
% Marges extérieures pour l'annotation
\usepackage[left=1.25cm, right=1.25cm,top = 1.5cm, bottom = 1.5cm]{geometry}
% Pas de marges pour la version finale
%\usepackage[left=1.5cm, right=3.5cm,top = 2.5cm, bottom = 2.5cm]{geometry}
% TODO : penser à remettr correctement les marges
%\setlength{\hoffset}{-18pt}
% \setlength{\oddsidemargin}{0cm}   % Marge gauche sur pages impaires
% \setlength{\evensidemargin}{4 cm}  % Marge gauche sur pages paires
% \setlength{\marginparwidth}{4cm} % Largeur de note dans la marge
% \setlength{\textwidth}{452pt}     % Largeur de la zone de texte (17cm)
% \setlength{\voffset}{-18pt}       % Bon pour DOS
% \setlength{\marginparsep}{4cm}    % Séparation de la marge
% \setlength{\topmargin}{0pt}       % Pas de marge en haut
% \setlength{\headheight}{13pt}     % Haut de page
% \setlength{\headsep}{10pt}        % Entre le haut de page et le texte
% \setlength{\footskip}{27pt}       % Bas de page + séparation
% \setlength{\textheight}{650pt}    % Hauteur de la zone de texte (25cm)
%


\usepackage{fancyhdr}
\pagestyle{fancy}
\usepackage{lastpage}
\renewcommand{\headrulewidth}{1pt}
\fancyhead[L]{CReal}
\fancyhead[R]{J. \textsc{Giet}}
\fancyfoot[C]{\thepage/\pageref{LastPage}}

%\renewcommand*\rmdefault{ppl}

%%%%%%%%%%%%%%%%%%%%%%%%%%%%%%%%%%%%%%%%%%%%
%%%%          Raccourcis usuels         %%%%
%%%%%%%%%%%%%%%%%%%%%%%%%%%%%%%%%%%%%%%%%%%%

%%%%%   Symboles mathématiques %%%%%%
\newcommand{\K}{\ensuremath\mathbb{K}}
\newcommand{\N}{\ensuremath\mathbb{N}}
\newcommand{\Z}{\ensuremath\mathbb{Z}}
\newcommand{\D}{\ensuremath\mathbb{D}}
\newcommand{\Q}{\ensuremath\mathbb{Q}}
\newcommand{\R}{\ensuremath\mathbb{R}}
\newcommand{\M}{\ensuremath\mathbb{M}}
\renewcommand{\L}{\ensuremath\mathbb{L}}
\newcommand{\U}{\ensuremath\mathbb{U}}
\newcommand{\C}{\ensuremath\mathbb{C}}
\newcommand{\E}{\ensuremath\mathbb{E}}
\newcommand{\V}{\ensuremath\mathbb{V}}
\newcommand{\B}{\texttt{\_B}}
\renewcommand{\O}{\ensuremath\mathcal{O}}
\renewcommand{\P}{\mathcal{P}}
\newcommand{\A}{\mathcal{A}}
\renewcommand{\a}{{\scriptscriptstyle\mathcal{A}}}
\renewcommand{\S}{\mathcal{S}}

%%%%% DEs raccourcis propres au projet
\newcommand{\reduc}{\textbf{reduc}}
\newcommand{\trans}{\textbf{trans}}
\newcommand{\guard}{\textbf{guard}}
\newcommand{\defi}{\text{def.}}

%%%%%  Parenthèses & autres %%%%%%
\newcommand{\po}{\left(} % grande parenthèse ouvrante
\newcommand{\pf}{\right)} % grande parenthèse fermante
\newcommand{\ao}{\left\lbrace}
\newcommand{\af}{\right\rbrace}
\newcommand{\dco}{\llbracket}
\newcommand{\dcf}{\rrbracket}

%%%% Raccourcis usuels   %%%%
\newcommand{\ie}{\textit{i.e. }}
\newcommand{\eg}{\textit{e.g. }}
\newcommand{\cf}{\textit{cf. }}
\newcommand{\ssi}{\text{ si et seulement si }}
\newcommand{\si}{\text{ si }}

%%%%  Pour le module math   %%%%
\newcommand{\et}{\text{~et }}
\newcommand{\ou}{\text{~ou }}
\newcommand{\car}{\text{~car }}
\newcommand{\avec}{\text{~avec }}

%%%%  Flêches et autres  %%%%
\renewcommand{\le}{\leqslant}
\renewcommand{\ge}{\geqslant}
\renewcommand{\leq}{\leqslant}
\renewcommand{\geq}{\geqslant}
\renewcommand{\emptyset}{\varnothing}
\newcommand{\la}{\leftarrow}
\newcommand{\xla}{\xleftarrow}
\newcommand{\ra}{\rightarrow}
\newcommand{\xra}{\xrightarrow}

\usepackage{xcolor}
\newcommand{\TODO}{{\color{red}\textbf{\textit{TODO}}}}



%%%% Pour la méta
% \newcommand{\TODO}{\textgreek{\textbf{\color{red}TODO}}}
% \newcommand{\Rq}{\textbf{\color{blue}Remarque(s)}}


\renewcommand{\listingscaption}{Code}
\renewcommand{\listoflistingscaption}{Table des fragments de code}
% Définition de l'envirronement Code pour le OCaml
\newenvironment{code}[1][ocaml]{%
\newgeometry{textwidth = 0.8\textwidth}
\VerbatimEnvironment\begin{minted}[
%linenos,
frame=lines,
framesep=2mm,
fontsize=\footnotesize,
mathescape,
autogobble]{#1}%
}{%
\end{minted}
\restoregeometry
}

\usepackage[nottoc, notlof, notlot]{tocbibind}

% Pour les numéros de section en chiffre romain
\renewcommand{\thesection}{\Roman{section}}

\title{CReal: Arithmétique en précision arbitraire}   % Mettre le titre içi!!
\author{Josselin \textsc{Giet}}  % Renseigner le nom de l'auteur
%\date{}    % Par défaut ,on ne met pas de date
%\date{}
% SOn commente cette ligne en dessus pour savoir les dates de compilation

\begin{document}
\maketitle

\section*{Introduction}
\TODO
%\tableofcontents

\section{Axioms}

%----------------------------------------------------------------------------
% II. function on integers
%----------------------------------------------------------------------------
\section{Functions on integers}

% I.1. power2
\textbf{II.1}
Cette question n'a pas présenté de grandes difficultés.
J'utilise une version itérative « naïve » de l'exponentielle pour
calculer $2^n$.
J'ai aussi rajouter un \texttt{ensures} de plus sur le résultat :
il est strictement positif.
Cela a été nécessaire dans la question \textbf{II.4} afin de montrer que
$2^\ell > 0$

\begin{listing}[H]
\begin{code}[ocaml]
	let i = ref 33 in incr i; !i
\end{code}
\end{listing}

\textbf{II.2}
Une fois donnée la définition de la fonction \texttt{power2}, la défnition de
\texttt{shift\_left} est immédiate.

\textbf{II.3}
Pour la division euclidienne \texttt{ediv\_mod}, j'ai dû faire une disjonction
de cas selon le signe du dividende \texttt{x}.
Chaque cas se fait de manière itérative : on maintient un compteur \texttt{d}
et un accumulateur \texttt{r} pour le reste.

\textbf{II.4}
De même que pour la question I.2, la fonction \texttt{shift\_right} est définie
sans problèmes en utilisant la divsion euclidienne de la fonction
précédente.

\textbf{II.5}
Pour la fonction \texttt{isqrt}, j'ai fait une version itérative proche de
celle dans l'exercice du cours à une exception près : la conition d'arrêt de
la boucle \texttt{while} se fait en testant l'inégalité pour \texttt{!res + 1}.

\section{Difficulty of Non-linear Arithmetic on Reals}

\textbf{III.6-12}
Pour ces questions, j'ai tout simplement les lemmes sous forme de
\textit{function lemmas} et les solveurs ont prouvé facilement ces lemmes.
La seule \textit{difficulté} rencontrée fût dans la question \textbf{III.9},
où j'ai dû rajouter un \texttt{assert} dans le corps de la fonction pour
montrer que $\sqrt{\B(2n)} = \B(n)$.

\textbf{III.13}

Dans \TODO il y a deux deux définitions equvalentes de \texttt{framing} :
\begin{align}
\forall x \in \R, p, n \in \Z,\  \nonumber
	& \texttt{framing}\ x\ p\ n \\
\Longleftrightarrow \ \
	& \left( p-1 \right) \cdot 4^{-n}
		< x <
		\left( p+1 \right) \cdot 4^{-n}
	\label{eq:fram-1} \\
\Longleftrightarrow \ \
	& |x-p\cdot 4^{-n}| < 4^{-n}
	\label{eq:fram-2}
\end{align}

La définition \ref{eq:fram-1} est préférable à \ref{eq:fram-2} car elle fait
intervenir moins de fonctions non-linéaires.
En effet, il n'y a pas de de valeurs absolues dans la première définition.

\section{Computational Real}

\textbf{IV.14}
Pour la fonction \texttt{round\_z\_over\_4}, j'ai prouvé les deux inégalités
séparément. Pour l'inégalité de gauche, la préuve se  base sur le fait que :
$$
	(z-2)\B(-1)
	= \frac{z-2}{4}
	< \floor*{ \frac{z - 2}{4} + 1 }
	= \floor*{ \frac{z+2}{4} }
$$
et pour celle de droite :
$$
\floor*{ \frac{z + 2}{4} }
	\leq \frac{z+2}{4}
	= (z + 2)\B(-1)
$$

Pour la fonction








































%\newpage
%\tableofcontents
%\listoflistings
%\listoffigures
%\listoftables
%\newpage


\end{document}
